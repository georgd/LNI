% Currently this document is written in German
% !TeX spellcheck = de_DE

%Ensure that all odl school LaTeX habits are remarked
\RequirePackage[l2tabu, orthodox]{nag}
%Neue deutsche Trennmuster
%Siehe http://www.ctan.org/pkg/dehyph-exptl und http://projekte.dante.de/Trennmuster/WebHome
%Nur für pdflatex, nicht für lualatex
\RequirePackage[ngerman=ngerman-x-latest]{hyphsubst}
\documentclass{lni}
% in Englisch stattdessen:
%\documentclass[english]{lni}







%% Some packages, no need to be adabted

% enable copy and paste of ligatures (e.g., in "workflow" and umlauts)
\usepackage{cmap}

%Überschrift des Literaturverzeichnisses
%Only in German
\iflnienglish
\else
\renewcommand{\refname}{Literaturverzeichnis}
\fi

%Enable input of umlauts using UTF-8.
\usepackage[utf8]{inputenc}

\usepackage{graphicx}

\usepackage{fancyhdr}

%Kopf- und Fußzeileneinstellungen
\fancypagestyle{lnifirstpage}{
% Löscht alle Kopf- und Fußzeileneinstellungen
\fancyhf{}

%Kopfzeile
\fancyhead[RO]{\small SOAP/XML ist nur ein mögliches Binding\linebreak%
Seminarbeitrag für Seminar \enquote{StuPro IoT-XT}
}
%Für den Herausgeber:
%\fancyhead[RO]{\small <Vorname Nachname [et. al.]> (Hrsg.): <Buchtitel>,\linebreak%
%Lecture Notes in Informatics (LNI), Gesellschaft für Informatik, Bonn <Jahr> \hspace{5pt} \thepage \hspace{0.05cm}}

%Linie unter Kopfzeile
\renewcommand{\headrulewidth}{0.4pt}
}

% Put in the short title (Kurztitle) here
\fancypagestyle{lni}{
\fancyhf{}
\fancyhead[RO]{\small SOAP/XML ist nur ein mögliches Binding \hspace{5pt}
\thepage \hspace{0.05cm}}
\fancyhead[LE]{\hspace{0.05cm}\small \thepage \hspace{5pt}
Vorname Nachname}
\renewcommand{\headrulewidth}{0.4pt}
}

%if lstlistings is used
%better approach: use the minted package - see https://en.wikibooks.org/wiki/LaTeX/Source_Code_Listings#The_minted_package
\usepackage{listings}

\iflnienglish
\usepackage[figurename=Fig., tablename=Tab., small]{caption}
\else
\usepackage[figurename=Abb., tablename=Tab., small]{caption}
\fi

% Listingname heißt nun List.
\renewcommand{\lstlistingname}{List.}

%for easy quotations: \enquote{text}
\usepackage{csquotes}

\usepackage[T1]{fontenc}

%enable margin kerning
\usepackage{microtype}

%for demonstration purposes only
\usepackage[math]{blindtext}

%tweak \url{...}
\usepackage{url}
%nicer // - solution by http://tex.stackexchange.com/a/98470/9075
\makeatletter
\def\Url@twoslashes{\mathchar`\/\@ifnextchar/{\kern-.2em}{}}
\g@addto@macro\UrlSpecials{\do\/{\Url@twoslashes}}
\makeatother
\urlstyle{same}
%improve wrapping of URLs - hint by http://tex.stackexchange.com/a/10419/9075
\makeatletter
\g@addto@macro{\UrlBreaks}{\UrlOrds}
\makeatother

%diagonal lines in a table - http://tex.stackexchange.com/questions/17745/diagonal-lines-in-table-cell
%slashbox is not available in texlive (due to licensing) and also gives bad results. Thus, we use diagbox
%\usepackage{diagbox}

\usepackage{booktabs}

%required for pdfcomment later
\usepackage{xcolor}

% new packages BEFORE hyperref
% See also http://tex.stackexchange.com/questions/1863/which-packages-should-be-loaded-after-hyperref-instead-of-before

%enable hyperref without colors and without bookmarks
\usepackage[
%pdfauthor={},
%pdfsubject={},
%pdftitle={},
%pdfkeywords={},
bookmarks=false,
breaklinks=true,
colorlinks=true,
linkcolor=black,
citecolor=black,
urlcolor=black,
pdfpagelayout=SinglePage,
pdfstartview=Fit
]{hyperref}
%enables correct jumping to figures when referencing
\usepackage[all]{hypcap}

%enable nice comments
\usepackage{pdfcomment}
\newcommand{\commentontext}[2]{\colorbox{yellow!60}{#1}\pdfcomment[color={0.234 0.867 0.211},hoffset=-6pt,voffset=10pt,opacity=0.5]{#2}}
\newcommand{\commentatside}[1]{\pdfcomment[color={0.045 0.278 0.643},icon=Note]{#1}}

%compatibality with TODO package
\newcommand{\todo}[1]{\commentatside{#1}}

%enable \cref{...} and \Cref{...} instead of \ref: Type of reference included in the link
\iflnienglish
\usepackage[capitalise,nameinlink]{cleveref}
%Nice formats for \cref
\crefname{section}{Sect.}{Sect.}
\Crefname{section}{Section}{Sections}
\crefname{figure}{Fig.}{Fig.}
\Crefname{figure}{Figure}{Figures}
\else
\usepackage[ngerman,capitalise,nameinlink]{cleveref}
\fi

%introduce \powerset - hint by http://matheplanet.com/matheplanet/nuke/html/viewtopic.php?topic=136492&post_id=997377
\DeclareFontFamily{U}{MnSymbolC}{}
\DeclareSymbolFont{MnSyC}{U}{MnSymbolC}{m}{n}
\DeclareFontShape{U}{MnSymbolC}{m}{n}{
    <-6>  MnSymbolC5
   <6-7>  MnSymbolC6
   <7-8>  MnSymbolC7
   <8-9>  MnSymbolC8
   <9-10> MnSymbolC9
  <10-12> MnSymbolC10
  <12->   MnSymbolC12%
}{}
\DeclareMathSymbol{\powerset}{\mathord}{MnSyC}{180}

%improve float placement
%source: http://people.cs.uu.nl/piet/floats/node1.html
%see also: http://tex.stackexchange.com/a/35130/9075
\renewcommand{\textfraction}{0.05}
\renewcommand{\topfraction}{0.95}
\renewcommand{\bottomfraction}{0.95}
\renewcommand{\floatpagefraction}{0.35}
\setcounter{totalnumber}{5}

% correct bad hyphenation here
\hyphenation{net-works semi-conduc-tor}


%%% Adapt to your needs from here

%Beginn der Seitenzählung für diesen Beitrag
\setcounter{page}{1}

\author{Vorname Nachname\footnote{Universität Stuttgart, IAAS, emailadresse@author}}

\title{SOAP/XML ist nur ein mögliches Binding}

\begin{document}
\maketitle

%hint by http://tex.stackexchange.com/a/30229/9075 and http://tex.stackexchange.com/a/247652/9075
\thispagestyle{lnifirstpage}
\pagestyle{lni}

\setcounter{footnote}{1}

\begin{abstract}
Kurze Zusammenfassung
\end{abstract}

\begin{keywords}
SOAP.
XML.
Binding.
SOA.
Web Services.
\end{keywords}

\section{Einleitung}
Dieses Template ist ein Schnellstart für Seminare.

\blindtext

\Cref{sec:lniconformance} zeigt die Einhaltung der Richtlinien durch einfachen Text.
Diesen Text durch eine eigene Gliederung ersetzen.

\section{Demonstration der Einhaltung der Richtlinien}
\label{sec:lniconformance}

\subsection{Literaturverzeichnis}
Der letzte Abschnitt zeigt ein beispielhaftes Literaturverzeichnis für Bücher mit einem Autor \cite{Ez10} und zwei AutorInnen \cite{AB00}, einem Beitrag in Proceedings mit drei AutorInnen \cite{ABC01}, einem Beitrag in einem LNI Band mit mehr als drei AutorInnen \cite{Az09}, zwei Bücher mit den jeweils selben vier AutorInnen im selben Erscheinungsjahr \cite{Wa14} und \cite{Wa14b}, ein Journal \cite{Gl06}, eine Website \cite{GI14} bzw.\ anderweitige Literatur ohne konkrete AutorInnenschaft \cite{XX14}.

Formatierung und Abkürzungen werden für die Referenzen \texttt{book}, \texttt{inbook}, \texttt{proceedings}, \texttt{inproceedings}, \texttt{article}, \texttt{online} und \texttt{misc} automatisch vorgenommen.
Mögliche Felder für Referenzen können der Beispieldatei \texttt{paper.bib} entnommen werden.
Andere Referenzen bzw.\ Felder müssen allenfalls nachträglich angepasst werden.

\subsection{Abbildungen}
\Cref{fig:demo} zeigt eine Abbildung.

\begin{figure}
  \centering
  Hier sollte die Graphik mittels \texttt{includegraphics} eingebunden werden.

  %\includegraphics[width=.8\textwidth]{filename}
  \caption{Demographik}
  \label{fig:demo}
\end{figure}

\subsection{Tabellen}
\Cref{tab:demo} zeigt eine Tabelle.

\begin{table}
\centering
\begin{tabular}{lll}
\toprule
Überschriftsebenen & Beispiel & Schriftgröße und -art \\
\midrule
Titel (linksbündig) & Der Titel \ldots & 14 pt, Fett\\
Überschrift 1 & 1 Einleitung & 12 pt, Fett\\
Überschrift 2 & 2.1 Titel & 10 pt, Fett\\
\bottomrule
\end{tabular}
\caption{Die Überschriftsarten}
\label{tab:demo}
\end{table}

\subsection{Programmcode}
Die LNI-Formatvorlage verlangt die Einrückung von Listings vom linken Rand.
In der \texttt{lni}-Dokumentenklasse ist dies für die \texttt{verbatim}-Umgebung realisiert.

\begin{verbatim}
public class Hello { 
    public static void main (String[] args) { 
        System.out.println("Hello World!"); 
    } 
} 
\end{verbatim}

Alternativ kann auch die \texttt{lstlisting}-Umgebung verwendet werden.

\Cref{L1} zeigt uns ein Beispiel, das mit Hilfe der \texttt{lstlisting}-Umgebung realisiert ist.

\lstset{basicstyle=\ttfamily}
\begin{lstlisting} [captionpos=b, caption=Beschreibung, label=L1, xleftmargin=0.5cm]
public class Hello { 
    public static void main (String[] args) { 
        System.out.println("Hello World!"); 
    } 
}
\end{lstlisting}

\subsection{Formeln und Gleichungen}

Die korrekte Einrückung und Nummerierung für Formeln ist bei den Umgebungen \texttt{equation} und \texttt{eqnarray} gewährleistet.

\begin{equation}
  1=4-3
\end{equation}
und
\begin{eqnarray}
  2=7-5\\
  3=2-1
\end{eqnarray}

%DO NOT USE \bibliographystyle - is is already done in lni.cls
%\bibliographystyle{lnig}

\bibliography{paper}

\end{document}
